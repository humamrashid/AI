% Homework, Chapter 7.
% Humam Rashid
% Spring 2020, CISC 7410X, Prof. Raphan

\documentclass{scrartcl}
\usepackage[a4paper, margin=1in]{geometry}
\usepackage[shortlabels]{enumitem}
\begin{document}

\begin{flushleft}
Humam Rashid\\
Spring 2020, CISC 7410X, Prof. Raphan\\
\underline{Homework, Questions from Chapters 7,}\\
\textit{Artificial Intelligence: A Modern Approach}, 3rd Edition
\end{flushleft}

\section*{Chapter 7}
\textbf{7.2} Given the premise, we can deduce the following:
\begin{enumerate}
    \item The unicorn being mythical implies it is immortal.
    \item The unicorn not being mythical implies it is a mortal mammal.
    \item From the above two statements, the unicorn is either immortal or a mammal, there is no
        other possibility.
    \item In either case (being immortal or a mammal), the unicorn is horned, and since these two
        cases are the only possibilities, the unicorn is always horned.
    \item The unicorn being horned implies it is magical, and since the unicorn is always horned,
        the unicorn is always magical.
\end{enumerate}
We cannot deduce deduce from this premise whether the unicorn is mythical or not. We can however
deduce that in all $2^5$ models, the unicorn is horned and it is magical.
\bigskip
\\
\textbf{7.3} Given the premise, we can deduce the following:

\end{document}

% EOF.
