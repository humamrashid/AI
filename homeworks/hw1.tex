% Homework #1
% Humam Rashid
% Spring 2020, CISC 7410X, Prof. Raphan

\documentclass{scrartcl}
\usepackage[utf8]{inputenc}
\usepackage[a4paper, margin=1in]{geometry}
\usepackage[shortlabels]{enumitem}
\begin{document}

\begin{flushleft}
Humam Rashid\\
Spring 2020, CISC 7410X, Prof. Raphan\\
\underline{Homework 1, Questions from Chapters 1 and 2,}\\
\textit{Artificial Intelligence: A Modern Approach}, 3rd Edition
\end{flushleft}

\section*{Chapter 1}
\textbf{1.1} Definitions:
\begin{enumerate}[(a)]
    \item intelligence: the ability to reason and make choices.
    \item artificial intelligence: the simulation of intelligent (i.e., human or animal-like)
        behavior and decision-making in computing machines; in most of this book however, the more
        narrower definition used is that of studying ``agent'' programs that can simulate
        ``acting rationally'' in a given environment.
    \item agent: a ``computing device'' (a machine run by one or more computers) that performs
        actions (using actuators) in response to feedback (acquired through sensors) from the
        environment.
    \item rationality: the property of intelligent beings to make the ``right decision'' based on
        current knowledge and a given performance measure.
    \item logical reasoning: any system of reasoning with sentences indicating truth values and
        working within the contraints of a set of definitions which are assumed to be true within
        the system.
\end{enumerate}
\textbf{1.4} Intractibility or undecidability of certain problems indicates one reason that
        simulating ``perfect rationality'' is in fact not possible in AI. However, limited forms of
        simulating rational behavior or rational decision-making is neither proven nor disproven by
        this issue. AI can still be useful with limited forms of simulating intelligent behavior.
\bigskip
\\
\textbf{1.10}
\bigskip
\\
\textbf{1.11}
\bigskip
\\
\textbf{1.13}
\section*{Chapter 2}
\textbf{2.1}
\bigskip
\\
\textbf{2.2}
\bigskip
\\
\textbf{2.3}

\end{document}

% EOF.
