% Homework #1
% Humam Rashid
% Spring 2020, CISC 7410X, Prof. Raphan

\documentclass{scrartcl}
\usepackage[a4paper, margin=1in]{geometry}
\usepackage[shortlabels]{enumitem}
\begin{document}

\begin{flushleft}
Humam Rashid\\
Spring 2020, CISC 7410X, Prof. Raphan\\
\underline{Homework 1, Questions from Chapters 1 and 2,}\\
\textit{Artificial Intelligence: A Modern Approach}, 3rd Edition
\end{flushleft}

\section*{Chapter 1}
\textbf{1.1} Definitions:
\begin{enumerate}[(a)]
    \item intelligence: the ability to reason and make choices.
    \item artificial intelligence: the simulation of intelligent (i.e., human or animal-like)
        behavior and decision-making in computing machines; in most of this book however, the more
        narrower definition used is that of studying ``agent'' programs that can simulate
        ``acting rationally'' in a given environment.
    \item agent: a ``computing device'' (a machine run by one or more computers) that performs
        actions (using actuators) in response to feedback (acquired through sensors) from the
        environment.
    \item rationality: the property of intelligent beings to make the ``right decision'' based on
        current knowledge and a given performance measure.
    \item logical reasoning: any system of reasoning with sentences indicating truth values and
        working within the constraints of a set of definitions which are assumed to be true within
        the system.
\end{enumerate}
\textbf{1.4} Intractability or undecidability of certain problems may indicate one reason that
simulating ``perfect rationality'' is in fact not possible in AI. However, limited forms of
simulating rational behavior or rational decision-making is neither proved nor disproved by this
issue. AI can still be useful with limited, more specialized forms of simulating intelligent
behavior.
\bigskip
\\
\textbf{1.10} AI is an interdisciplinary field combining information from several different science
(e.g., neuroscience, cognitive science, physics), engineering (e.g., control theory, robotics) and
humanities (e.g., economics, psychology) research areas. The field of AI, as it is known to us now,
combines elements of all these fields in its theoretical form. In terms of application areas, an
even wider range of fields and research areas could be included within AI studies; one need only
look at the various applications of AI in everyday life today to see this, with examples including
arts, medicine, manufacturing, transportation, etc. Of course, computer science, mathematics, logic
and philosophy continue to play central roles in the development of AI. I can't help but compare AI
(at least in a historical sense) to alchemy given the varied perspectives on its definitions, wide
spectrum of information sources and a clearer definition of certain expected or desired end results
rather than an agreed upon foundation and purpose. It seems to me that the alchemy to chemistry
transition is starting to take place in AI with the focus on rational agents (as per what the
authors of this book allege).
\bigskip
\\
\textbf{1.11} I do think it is true that computers are limited to their programming, but simulating
intelligent behavior in computing machines is not negated by that fact, especially since certain
techniques (such as randomization and machine learning) can lead to results which are unpredictable
(at least from the perspective of the programmers).
\bigskip
\\
\textbf{1.13} I do not think the latter statement is true, it has not been shown that thoughts are
governed through the laws of physics. The laws of physics aren't completely known or understood to
begin with and even if our thoughts and actions are generally constrained within our current
understanding of physics, intelligence is not defined in terms of the physical makeup of intelligent
beings.
\section*{Chapter 2}
\textbf{2.1} Rationality of Vacuum-Cleaner Agent Functions:
\begin{enumerate}[(a)]
    \item
\end{enumerate}
%\bigskip
%\\
%\textbf{2.2}
%\bigskip
%\\
%\textbf{2.3}

\end{document}

% EOF.
